\documentclass[12pt,a4paper]{article}
% Paquetes para idioma español, codificación y mejor presentación
\usepackage[utf8]{inputenc}
\usepackage[T1]{fontenc}
\usepackage[spanish]{babel}
\usepackage{geometry}
\geometry{margin=2.5cm}
\usepackage{setspace}
\setstretch{1.25}

% Datos para portada personalizada
\title{\textbf{Estrategias de diversificación global a través del deporte}\\[0.5em]
\large Estudio de casos City Football Group y Red Bull}

\author{
\textbf{Grupo Cookies}\\[0.5em]
    Briggs Emma, Briggs Sofía, Constantinidi Sofía, Nallib Delfina, Pedrosa Guadalupe\\[3em]
\\
\textit{Universidad Nacional de Cuyo} \\
\textit{Facultad de Ingeniería} \\
\textit{Asignatura: Técnica y Herramientas Modernas}
}

\date{June 2025}

\begin{document}

% Portada
\maketitle
\thispagestyle{empty}
\newpage

% Parte 1
\section{Parte 1}

\subsection{Introducción}

En las últimas décadas, el deporte profesional ha evolucionado de un espectáculo local a una plataforma clave en la economía global, impulsada por la globalización de marcas, capitales e identidades culturales. Los clubes deportivos, antes anclados en la comunidad y la competencia local, se han transformado en estructuras multiclub o multideporte gestionadas con criterios corporativos, enfocadas en el crecimiento transnacional y la consolidación de marca.

Este proceso refleja la apropiación del deporte por actores empresariales que lo integran en estrategias de crecimiento a través de adquisiciones, patrocinios y desarrollo de infraestructura, articulando consumo, identidad y mercado. El presente trabajo analiza cómo las empresas utilizan el deporte profesional para su crecimiento internacional y fortalecimiento de marca, mediante el estudio de dos modelos de gestión contrastantes para identificar sus motivaciones, ventajas (económicas, logísticas, simbólicas) e implicaciones para el futuro del deporte como fenómeno cultural y económico.

\subsection{Contextualización}

En el contexto de la globalización deportiva, empresas como City Football Group (CFG) y Red Bull han transformado el deporte en una herramienta estratégica para el crecimiento internacional, la diversificación de activos y la consolidación de marca. Este artículo analiza estos dos casos, cuyos enfoques innovadores y exitosos, aunque diferenciados, configuran un nuevo modelo de gestión deportiva transnacional.

\subsubsection{City Football Group (CFG)}

Establecido en 2013 con capital de Abu Dhabi, City Football Group (CFG) lidera el modelo multiclub mediante la adquisición o participación accionaria en clubes ubicados en diversas regiones geográficas, incluyendo Europa (Reino Unido, España), Asia (Japón), América (Estados Unidos, Uruguay) y Oceanía (Australia).  Su red global comparte identidad visual, metodologías de juego, sistemas de scouting y tecnologías de análisis de datos, optimizando recursos, desarrollando talento con perspectiva internacional y consolidando una marca unificada en mercados diversos mientras optimiza la eficiencia operativa.

\subsubsection{Red Bull}

Desde los años 90, Red Bull, conocida por sus bebidas energéticas, ha integrado el deporte como pilar de su estrategia de marca, incursionando en disciplinas como Fórmula 1, deportes extremos y fútbol. A diferencia de CFG, Red Bull adapta cada proyecto deportivo a su identidad de marca, asociada a lo audaz, joven y competitivo. Sus equipos de fútbol en Austria, Alemania, Brasil y Estados Unidos, junto con su escudería de Fórmula 1, funcionan como extensiones de la marca, complementadas por eventos, medios audiovisuales y plataformas (Red Bull TV).

\subsection{Comparación preliminar}

Ambos modelos evidencian que el deporte trasciende el rendimiento competitivo, funcionando como una plataforma multifuncional que genera valor financiero, simbólico y cultural. CFG apuesta por la integración global y la estandarización, mientras Red Bull privilegia la flexibilidad y la diferenciación local. Esta comparación muestra cómo distintas lógicas empresariales convergen en el uso del deporte para expandirse globalmente, conectando mercados, construyendo narrativas culturales y consolidando marcas en un entorno globalizado.

La profesionalización en la gestión deportiva, el empleo de tecnologías avanzadas (análisis de datos en CFG, producción mediática en Red Bull) y la adaptación a regulaciones locales son factores clave que sostienen estos modelos. No obstante, enfrentan desafíos: CFG corre el riesgo de diluir identidades locales por su uniformidad global, mientras Red Bull puede limitar su integración con comunidades tradicionales debido a su imagen juvenil y audaz. Su reto es equilibrar estandarización y adaptación para conservar relevancia cultural y respaldo comunitario mientras consolidan su expansión.

\subsection{Marco teórico}

La consolidación de estructuras empresariales en el deporte profesional supera el éxito competitivo, integrando estrategias de marketing, posicionamiento global, inversión y gestión de activos simbólicos. La diversificación deportiva va más allá del espectáculo: las empresas buscan acceder a mercados y fidelizar audiencias mediante conexiones emocionales profundas, construyendo ecosistemas comerciales que amplifican su alcance y fortalecen su posición internacional.

Económicamente, estos modelos generan economías de escala, diversifican ingresos y valorizan activos con mayor eficiencia. Logísticamente, optimizan el flujo de talento, tecnología y conocimiento entre sus unidades, incrementando la eficiencia organizacional. Simbólicamente, consolidan narrativas de marca que trascienden fronteras, fortaleciendo el reconocimiento global.

Sin embargo, estas transformaciones impactan el deporte tradicional, redefiniendo su carácter local y comunitario, modificando la pertenencia de los seguidores, las jerarquías competitivas y las dinámicas publicitarias. También generan tensiones sobre gobernanza, equidad y autenticidad cultural, poniendo en cuestión valores históricos del deporte.

Este artículo examina cómo la lógica empresarial de expansión y diversificación ha reformulado la gestión del deporte profesional, analizando cómo modelos corporativos, con sus recursos y particularidades, configuran un nuevo ecosistema deportivo global. Este paradigma emergente entrelaza rendimiento deportivo con estrategias financieras, de marca y estructuras transnacionales.

La integración de tecnologías avanzadas potencia la eficiencia y el alcance, aunque el equilibrio entre estandarización global y preservación de identidades locales sigue siendo clave para mantener la conexión con los aficionados.

En este contexto, el presente trabajo se propone responder la siguiente pregunta:

¿Cómo se expresa la globalización en los nuevos modelos de gestión deportiva?


% Parte 2
\section{Parte 2: Materiales y métodos}

Para esta investigación se realizó una exhaustiva revisión bibliográfica de artículos académicos, informes y documentos oficiales relacionados con la globalización y gestión en la industria deportiva. La recopilación, organización y citación de las fuentes se realizó mediante el uso de Zotero, un gestor bibliográfico que facilitó la administración ordenada y eficiente de la bibliografía consultada. Se implementó un análisis cualitativo del discurso empresarial, enfocándose en aspectos como la construcción de marca y el storytelling corporativo. Complementariamente, se llevó a cabo un análisis cuantitativo para evaluar el crecimiento económico y el impacto financiero de los modelos de diversificación y expansión global estudiados.

Entre los artículos más relevantes que sustentan este trabajo se encuentran “Innovations in Sports Industry: Trends and Transformations”, “Networks of International Football: Communities, Evolution and Globalization of the Game”, “Network Strategy and Sport: The Case of City Football Group”, “Red Bull’s Sports Empire” y “Red Bull’s Multi-Club Ownership Model”. Estos estudios permiten abordar el fenómeno desde múltiples ángulos: estructurales, estratégicos, económicos y simbólicos.

También se consultaron trabajos sobre gestión internacional como “Sport and International Management” y análisis globales como “The Impact of Globalization on the Development of the Sports Industry” y “The Internationalisation of City Football Group”. A su vez, se utilizaron informes económicos, estadísticas de mercado (valorización de clubes, ingresos por merchandising, audiencias globales) y documentos oficiales de organismos como FIFA, UEFA y el Foro Económico Mundial.


% Parte 3
\section{Parte 3: Interpretación de la información y resultados}
Los resultados evidencian una transformación significativa de los clubes deportivos tradicionales, que ahora funcionan como plataformas de marca global. Este cambio se manifiesta en la creación de sinergias entre equipos, disciplinas deportivas y mercados diversos, aprovechando herramientas como el scouting internacional, estrategias de patrocinio integradas y la segmentación precisa del público objetivo.
Además, se observa el desarrollo de nuevas formas de generar valor, que trascienden el rendimiento deportivo para abarcar dimensiones financieras y simbólicas. Así, los clubes no solo compiten en lo deportivo, sino que también construyen capital de marca y conexiones culturales que fortalecen su posicionamiento global.


\subsection{Análisis por caso}


\subsubsection{Red Bull}

Ha consolidado una red de clubes de fútbol en Austria, Alemania, Brasil y Estados Unidos, operando en ligas de distintos niveles competitivos, lo que ha impulsado un incremento del 60\% en su audiencia digital global entre 2022 y 2025. Su escudería de Fórmula 1, con más de 70 victorias desde 2010, genera un impacto mediático que refuerza su imagen de marca asociada a lo juvenil y competitivo. La estrategia incluye Red Bull TV, con más de 5 millones de suscriptores, y eventos anuales que atraen a más de 1 millón de asistentes presenciales, consolidando una plataforma mediática que amplifica su alcance global. Esta diversificación genera ingresos estimados en 300 millones de dólares anuales solo en actividades deportivas, según análisis del sector.

\subsubsection{City Football Group}

Fundado en 2013, gestiona clubes en cinco continentes, con un valor corporativo que ha crecido más del 400\% desde su creación. Ha invertido aproximadamente 500 millones de dólares en los últimos cinco años en infraestructura deportiva y campañas de marketing, gestionando un ecosistema de más de 1,200 jugadores y staff técnico. Su uso de tecnología de análisis de datos ha incrementado un 35\% el volumen anual de transferencias internas de jugadores y promueve unos 50 jugadores juveniles al año desde sus academias, reduciendo un 20\% los costos operativos en scouting, según informes especializados.

\subsection{Comparación entre modelos}

Red Bull y CFG utilizan el deporte como activo estratégico, pero sus enfoques difieren significativamente. Red Bull prioriza una narrativa mediática transversal, vinculada a un estilo de vida activo, con una red de clubes en cuatro países y una escudería de Fórmula 1 que genera un impacto global, respaldado por 5 millones de suscriptores en Red Bull TV y eventos con más de 1 millón de asistentes anuales. En contraste, CFG apuesta por una red interconectada de clubes en 12 ligas profesionales, optimizando recursos mediante tecnología avanzada, con una inversión de 500 millones de dólares en cinco años y una reducción del 20\% en costos de scouting. Mientras Red Bull diversifica disciplinas para amplificar su marca (fútbol, Fórmula 1, deportes extremos), CFG centra su estrategia en la estandarización operativa, movilizando talento y generando sinergias entre clubes.

Ambos modelos convierten el deporte en una plataforma que combina valor financiero, cultural y mediático. Red Bull logra un crecimiento de audiencia digital del 60\% en tres años, mientras CFG incrementa su valor corporativo en más del 400\% desde 2013. Sin embargo, sus retos difieren: Red Bull enfrenta riesgos de dependencia de su imagen juvenil, potencialmente limitando su arraigo en comunidades deportivas tradicionales, mientras CFG debe mitigar la posible erosión de identidades locales debido a su estandarización global. Ambos buscan equilibrar estas dinámicas para sostener su influencia en mercados globales, integrando tecnología, marketing y gestión deportiva.


% Parte 4
\section{Parte 4: Conclusiones}

\subsection{Implicancias y Perspectivas Futuras}

El análisis de los modelos de Red Bull y City Football Group evidencia la consolidación de un nuevo paradigma de gestión deportiva que integra dimensiones empresariales, culturales y económicas. El deporte se posiciona como una herramienta estratégica para la expansión global de marcas, generando capital económico y simbólico. 

Estos casos desafían la noción tradicional del club deportivo como entidad comunitaria, planteando tensiones sobre la identidad, la gobernanza y la equidad en el deporte. La transformación de los clubes en plataformas globales requiere equilibrar rentabilidad con los valores culturales que históricamente han definido al deporte. 
Este modelo invita a reflexionar sobre el futuro del deporte como fenómeno cultural y económico, y sobre la posible replicación de estas estrategias en otras industrias, marcando una tendencia hacia una gestión empresarial más integrada en un contexto globalizado.

\end{document}














