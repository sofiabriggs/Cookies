\documentclass[a4paper,11pt]{article}
\usepackage[utf8]{inputenc}
\usepackage[margin=2.5cm]{geometry}
\usepackage{enumitem}
\usepackage{hyperref}

\title{\textbf{Currículum Vitae Grupal}}
\author{Grupo Cookies}
\date{Junio 2025}

\begin{document}

\maketitle

\noindent\textbf{Contacto:} UNCuyo - Facultad de Ingeniería, Mendoza, Argentina\\
\textbf{Email:} ingenieria404@uncuyo.edu.ar\\
\textbf{GitHub:} \href{https://github.com/sofiabriggs/Cookies}{github.com/sofiabriggs/Cookies}

\vspace{0.5cm}
\section*{¿Quiénes somos?}
Somos un grupo de cinco estudiantes de Ingeniería Industrial. Nos juntamos para trabajar en proyectos de la facultad y cosas que nos parecen interesantes y polémicas del mundo actual. Buscamos mostrar lo que hacemos.

\section*{Integrantes}

\begin{itemize}[leftmargin=*]
  \item \textbf{Delfina Nallib}: planificación, economía, gestión de proyectos.
  \item \textbf{Sofia Constantinidi}: simulación, programación, modelado matemático.
  \item \textbf{Sofia Briggs}: diseño CAD, análisis estructural, documentación técnica.
  \item \textbf{Emma Briggs}: automatización, sensores, control.
  \item \textbf{Guadalupe Pedrosa}: datos, visualización, IA.
\end{itemize}

\section*{Formación}
\textbf{Ingeniería Industrial} \\
Universidad Nacional de Cuyo \\
2022 - Actualidad

\section*{Proyectos académicos}

\begin{itemize}[leftmargin=*]
  \item Sistema de almacenamiento de CO líquido para protección eléctrica.
  \item Optimización de procesos productivos con simulación en Arena.
  \item Automatización de línea de montaje con PLC.
  \item Diseño de panel solar con sensores IoT y visualización web.
  \item Relevamiento y mejora de procesos en una PYME local.
\end{itemize}

\section*{Herramientas que usamos}

\begin{itemize}[leftmargin=*]
  \item \textbf{Lenguajes:} Python, MATLAB, algo de R, algo de C++.
  \item \textbf{Software:} Excel (avanzado), AutoCAD, SolidWorks, Arena, MS Project.
  \item \textbf{Extras:} análisis de datos, simulación, control estadístico, Lean.
\end{itemize}

\section*{Idiomas}
\begin{itemize}[leftmargin=*]
  \item Español (nativo)
  \item Inglés (intermedio/avanzado) – buen nivel técnico para papers y docs.
\end{itemize}

\section*{Cosas que nos interesan}
Tecnología aplicada, resolver problemas reales, datos, sostenibilidad.


\end{document}