\documentclass[12pt]{article}
\usepackage[utf8]{inputenc}
\usepackage[spanish]{babel}
\usepackage{amsmath}
\usepackage{graphicx}
\usepackage{hyperref}
\usepackage{enumitem}
\usepackage{fancyhdr}
\usepackage{geometry}
\geometry{margin=2.5cm}

\pagestyle{fancy}
\fancyhf{}
\rhead{Junio 2025}
\lhead{Grupo Cookies}
\cfoot{\thepage}

\title{\textbf{Informe: Integración de SIMUL8}}
\author{Grupo Cookies \\
Carrera: Ingeniería Industrial - 4to año \\
Facultad de Ingeniería, Universidad Nacional de Cuyo}
\date{Junio 2025}

\begin{document}

\maketitle
\hrule
\vspace{1em}
\section*{Materia: Técnicas y Herramientas Modernas}
\vspace{1em}
\hrule

\section{Introducción}

En el contexto de la simulación de procesos productivos y de servicios, \textbf{SIMUL8} se presenta como una herramienta gráfica poderosa y accesible para modelar sistemas discretos. Este informe tiene como objetivo presentar los conceptos fundamentales de construcción de modelos en SIMUL8, sus capacidades básicas y su aplicación en escenarios reales como un restaurante.

\section{Construcción básica de una cadena en SIMUL8}

La construcción de un modelo en SIMUL8 se basa en conectar \textbf{objetos gráficos} que representan elementos del sistema:

\begin{itemize}
    \item \textbf{Work Entry Point}: punto de entrada de las entidades (clientes, productos, etc.).
    \item \textbf{Work Centers}: representan estaciones de trabajo o procesos donde se consume tiempo y recursos.
    \item \textbf{Queues}: colas donde se almacenan las entidades cuando el centro de trabajo está ocupado.
    \item \textbf{Work Exit Point}: representa la salida del sistema.
\end{itemize}

El flujo básico de una cadena es:

\textbf{Entrada → Cola → Proceso → Salida}

Cada componente se conecta mediante flechas que representan la dirección del flujo de las entidades.

\section{Capacidad básica}

La \textbf{capacidad} en SIMUL8 puede referirse a:

\begin{itemize}
    \item La \textbf{cantidad de recursos} disponibles en un Work Center (por ejemplo, número de empleados, máquinas).
    \item El \textbf{tamaño máximo} de una cola (por defecto es infinita, pero se puede limitar).
    \item El \textbf{número de entidades} que pueden ser procesadas simultáneamente.
\end{itemize}

Este parámetro es clave para simular cuellos de botella o restricciones operativas reales.

\section{Tiempo de vida y distribución de probabilidad}

Cada Work Center puede tener asignado un \textbf{tiempo de procesamiento}, que puede definirse como:

\begin{itemize}
    \item Fijo (por ejemplo, 5 minutos por cliente).
    \item Distribución estadística:
    \begin{itemize}
        \item \textbf{Exponencial} (tiempos entre llegadas o servicios en colas).
        \item \textbf{Normal} (cuando se espera una variabilidad simétrica).
        \item \textbf{Uniforme}, \textbf{Triangular}, entre otras.
    \end{itemize}
\end{itemize}

Estas distribuciones permiten reflejar la \textbf{variabilidad real} de los procesos, algo que sería imposible modelar sólo con promedios.

El \textbf{tiempo de vida} puede entenderse como el tiempo que una entidad pasa dentro del sistema completo, desde la entrada hasta la salida.

\section{Routing In y Routing Out}

\subsection*{Routing In}

Define \textbf{cómo una entidad elige un Work Center} al cual ingresar desde una cola o punto anterior. Puede ser:

\begin{itemize}
    \item \textbf{Cualquiera disponible} (por defecto).
    \item Por \textbf{prioridad}.
    \item Por \textbf{mínimo contenido} (la cola menos llena).
    \item \textbf{Condicional}, usando lógica de decisiones.
\end{itemize}

\subsection*{Routing Out}

Define \textbf{a dónde se dirige una entidad} después de ser procesada:

\begin{itemize}
    \item Al siguiente Work Center disponible.
    \item Según probabilidades.
    \item Por condiciones o atributos.
\end{itemize}

Esto permite modelar flujos complejos, decisiones lógicas o incluso fallas/retrabajos.

\section{Aplicación: Caso de un Restaurante}

Un restaurante es un caso clásico para simular en SIMUL8, por su \textbf{flujo estructurado pero variable}. Ejemplo de elementos a modelar:

\begin{itemize}
    \item \textbf{Entrada de clientes} (Work Entry Point), según una distribución de llegada (ej: exponencial).
    \item \textbf{Cola para ser atendido}.
    \item \textbf{Asignación de mesa / camarero (Work Center)}.
    \item \textbf{Cocina} (otro Work Center con recursos limitados y tiempos variables).
    \item \textbf{Entrega de pedido y consumo} (se puede modelar como otro proceso).
    \item \textbf{Pago y salida} (Work Exit Point).
\end{itemize}

\subsection*{Elementos importantes a considerar}

\begin{itemize}
    \item Capacidad limitada de mesas.
    \item Variabilidad en el tiempo de preparación.
    \item Posibilidad de abandonar si la espera es muy larga (renuncia o \textit{balking}).
    \item Diferenciación por tipo de cliente o plato (atributos).
\end{itemize}

\subsection*{Profundizando en la simulación del restaurante}

Para modelar de forma realista un restaurante en SIMUL8, se pueden seguir estos pasos:

\begin{enumerate}
    \item \textbf{Definir tipos de clientes} con distintos comportamientos (por ejemplo: almuerzo rápido vs. cena completa).
    \item \textbf{Asignar atributos} a las entidades, como tipo de comida pedida, que influirá en el tiempo de cocina.
    \item \textbf{Crear distintos Work Centers} para cada etapa: recepción, asignación de mesa, cocina, entrega, pago.
    \item \textbf{Usar lógicas condicionales} para direccionar el flujo según si hay disponibilidad o no (por ejemplo: si no hay mesas, esperar en cola; si espera > X tiempo, abandonar).
    \item \textbf{Implementar horarios de mayor demanda} (picos de llegadas al mediodía o noche).
    \item \textbf{Configurar reportes} para analizar tiempo promedio en el sistema, utilización de recursos, cantidad de clientes perdidos, etc.
\end{enumerate}

Este tipo de simulación permite evaluar decisiones como:

\begin{itemize}
    \item Contratar más camareros o cocineros.
    \item Optimizar el \textit{layout} para reducir tiempos de espera.
    \item Diseñar una carta que balancee la carga de trabajo en cocina.
\end{itemize}

\section{Conclusiones}

SIMUL8 permite al ingeniero industrial construir modelos flexibles, visuales y estadísticamente robustos de sistemas reales. 



\end{document}